\documentclass[12pt]{article}
\usepackage{amssymb}
\usepackage[dvips]{graphicx}
%%\usepackage[cp850]{inputenc}
\usepackage[dvips]{graphicx}
\usepackage{multirow}
\usepackage{colortbl}
\oddsidemargin -0cm \textwidth 16cm \topmargin 0cm \textheight
22cm
\renewcommand{\baselinestretch}{1.2}
\newcommand{\func}{f: C \subseteq \mathbb{R}^{n}\longrightarrow \;\mathbb{R}}
\newcommand{\fun}{f: S \subseteq \mathbb{R}^{n}\longrightarrow \;\mathbb{R}}
\newtheorem{definition}{Definici\'on}[section]
\newtheorem{proposition}{Proposici\'on}[section]
\newtheorem{theorem}{Teorema}[section]
\newtheorem{example}{Ejemplo}[section]
\newtheorem{corollary}{Corolario}[section]
\newtheorem{lemma}{Lema}[section]
\newtheorem{remark}{Observaci\'on}[section]
\usepackage[T1]{fontenc}
\usepackage[utf8]{inputenc}
\usepackage[spanish]{babel}


\title{Problemas Nerea}
\date{} 

\begin{document}
\maketitle

\section*{Problema 1}
Se pide resolver la siguiente tabla de decisión.

\begin{tabular}{c|c|c|c|}
\multicolumn{4}{r}{Estados de la naturaleza} \\
\cline{2-4}
\multirow{3}{*}{Alternativas} & & e1 & e2 \\
\cline{2-4}
 & a1 & 3.000 & -1.000\\
\cline{2-4}
 & a2 & 2.000 & 2.000 \\
\cline{2-4}
\end{tabular}

Se debe resolver con cada uno de los métodos o funciones individuales de Incertidumbre por separado (tanto en situación favorable como desfavorable)


\section*{Problema 2}


Pepe es un estudiante que estudia un grado superior en su pueblo natal. Se plantea si cambiarse a un instituto de Sevilla, donde hay mayor cantidad de empresas, en las cuales espera obtener trabajo. Mudarse a Sevilla, le reportaría un coste de 3.000 euros. Sin embargo, si obtiene trabajo (suponiendo que la duración mínima fuera un año, con sueldo de 2000 euros), obtendría un beneficio de 24.000 euros.
Por el contrario, si no se cambia de instituto, no tendría ningún gasto.\\

Pepe quiere obtener el mayor beneficio de su decisión. Además, en caso de empate, quisiera tener más en cuenta el coste de oportunidad que tienen sus decisiones.

Las dos alternativas posibles para Pepe son:\\

\begin{itemize}

\item a1= Realizar un cambio de instituto
\item a2= Quedarse en el instituto en el que esta

\end{itemize}


Los dos estados de la naturaleza posibles son:\\

\begin{itemize}

\item a1= Ser contratado
\item a2= No ser contratado

\end{itemize}

La tabla de decisión para este problema seria:

\begin{tabular}{c|c|c|c|}
\multicolumn{4}{r}{Estados de la naturaleza} \\
\cline{2-4}
\multirow{3}{*}{Alternativas} & & Le contraten & No le contraten \\
\cline{2-4}
 & Cambiarse & 21.000 & -3.000\\
\cline{2-4}
 & No cambiarse & -24.000 & 0 \\
\cline{2-4}
\end{tabular}





\end{document}

%https://manualdelatex.com/tutoriales/listas-y-enumeraciones
% https://manualdelatex.com/tutoriales/tablas
% https://latexlive.files.wordpress.com/2009/04/tablas.pdf
% https://runebook.dev/es/docs/latex/_005ccline
% https://www.fing.edu.uy/~canale/latex.pdf